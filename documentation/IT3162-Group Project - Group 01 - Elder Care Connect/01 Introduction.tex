\chapter{Introduction}
Elderly individuals who live alone face significant health and safety challenges, including medical emergencies, falls, and feelings of isolation. Due to the lack of immediate assistance, these incidents can lead to delayed medical attention and complications. Furthermore, isolation can negatively affect their mental and emotional health. The absence of regular check-ins and health tracking increases the risk of undetected health deterioration.\\

The project aims to develop a Progressive Web Application (PWA) designed to enhance the safety, health monitoring, and social well-being of elderly individuals living alone. Through tracking real-time health (using APIs like Google Fit), the app will ensure caregivers/family members can monitor their health status from a distance.\\

The ElderCare Connect will function across multiple devices (desktops, smartphones, and tablets) with native app-like experiences, providing real-time health monitoring, emergency alerts, and social engagement features. Push notifications will ensure immediate attention during emergencies. Messaging features will help elderly connected with family, friends, and caregivers, mitigating loneliness and providing emotional support with PWA advantages. 
\section{Background}

In recent decades, smartphones have significantly changed the way we live. In terms of the health sector, elderly individuals now have greater access to health information and can enhance the convenience of managing their medical care through mobile applications. Several studies have shown the positive impact of mobile health interventions in managing lifestyle factors such as improving diet, quitting smoking, and increasing physical activity, as well as controlling chronic conditions like diabetes and hypertension. This change has also influenced the relationship between patients and their general practitioners.\\

Moreover, there has been increasing concern regarding the rising number of elderly individuals, particularly those living alone due to the population aging across the globe. This demographic group faces great challenges, including high health risks, social disconnectedness, and slow responses in case of emergencies that may endanger their lives. Considering that such a severe future scenario would likely occur much faster than a call for help or even faster than some normal medical conditions, there is a need for solutions that can protect the elderly, particularly those without nearby family members, from possible future threats.
\section{Problem Statement}
Elderly individuals who live alone often face significant challenges. In the event of an emergency, such as a fall or a sudden health complication, immediate assistance may not be available, potentially leading to severe consequences. Falls, in particular, can result in serious injuries, while unmanaged chronic conditions like diabetes require continuous monitoring to prevent critical health issues.\\

Social isolation is another major concern. Limited interaction with others can lead to feelings of loneliness and depression, which may negatively impact cognitive function and memory. As a result, some elderly individuals may neglect essential healthcare needs, such as attending medical appointments or adhering to prescribed medications.\\

While various mobile applications are available to assist older adults, they often lack comprehensive functionality. Many are designed for specific tasks, such as health monitoring, without integrating other essential features like emergency assistance or family communication. Additionally, these applications are frequently challenging to navigate, with small buttons, complex interfaces, and unclear instructions, making them less accessible for elderly users.\\

That's why we need a better app that is made just for older people who live alone. It needs to watch their health, call for help when they need it, and let them talk to family and friends.\\

ElderCare Connect will be a solution for this. It will use special sensors to watch their health and send alerts if something is wrong. It will have big, easy-to-see buttons and words. It will work even if they don't have internet all the time.\\
By making this app, we can help older people live safer, healthier, and happier lives. Their families can also feel better knowing they are okay.
\subsection{Main Objective}

Our primary objective is to develop Elder Care Connect, a specialized application designed to support independent elderly individuals. The application aims to enhance safety protocols, facilitate health monitoring, and promote social connectivity for seniors living alone.\\

The Elder Care Connect application will incorporate the following key functionalities:


\subsection{Scope of the Study}

This section delineates the parameters and limitations of the ElderCare Connect project, establishing clear boundaries for development and implementation.

\subsubsection*{Target Demographic:}
\begin{itemize}
    \item The application is designed for semi-independent elderly individuals requiring minimal assistance with safety and health monitoring.
    \item Users should possess basic technological literacy for interaction with mobile devices and tablets.
    \item The solution is not intended for individuals requiring intensive medical supervision or residing in assisted living facilities.
\end{itemize}

\subsubsection*{Application Functionality:}
\begin{itemize}
    \item Health monitoring through wearable device integration.
    \item Emergency alert system for crisis situations.
    \item Social connectivity tools to mitigate isolation.
    \item Cross-platform compatibility for mobile devices, tablets, and desktop computers.
    \item Offline functionality during connectivity interruptions.
    \item Secure health data management and storage.
\end{itemize}

\subsubsection*{Technical Implementation:}
\begin{itemize}
    \item Development utilizing current programming languages and frameworks.
    \item Integration with commercially available wearable biometric devices.
    \item Cloud-based data storage with appropriate security protocols.
    \item Compliance with relevant data protection and privacy regulations.
\end{itemize}

\subsubsection*{Geographic Implementation:}
\begin{itemize}
    \item Initial deployment in a controlled test region for validation purposes.
    \item Multilingual and multicultural adaptability for broader implementation.
\end{itemize}

\subsubsection*{Project Timeline:}
\begin{itemize}
    \item Structured development schedule with defined milestones.
    \item Focus on core functionality for initial release with planned feature expansion in subsequent iterations.
\end{itemize}

\subsubsection*{Exclusions:}
\begin{itemize}
    \item Development of proprietary medical hardware devices.
    \item Direct provision of medical services.
\end{itemize}
\section{Significance of the Study}
ElderCare Connect is important because it can make a big difference in the lives of older people, their families, and the community. It can help older people stay healthy, safe, and connected, and it can give their families peace of mind.\\

\textbf{How It Helps Older People:}
\begin{itemize}
    \item Better Life: It helps them live a better life by watching their health, getting help fast when they need it, and talking to family and friends.
    \item Less Time in the Hospital: It can keep them from having to go to the hospital by finding problems early and helping them get help fast.
    \item Happier Feelings: It helps them feel less lonely and more connected, which makes them happier.
    \item Stay Independent: It helps them stay in their own homes longer because they have the support they need to stay safe and healthy.
\end{itemize}
\textbf{How It Helps Families:}
\begin{itemize}
    \item Peace of Mind: It gives families peace of mind knowing that their loved ones are safe and healthy.
    \item Fast Help: It makes sure they get help fast if something goes wrong.
    \item Good Communication: It helps them talk to their loved ones and stay connected.
    \item Less Worry: It takes some of the worry away because they know their loved ones have the tools they need to stay safe.
\end{itemize}
\textbf{How It Helps the Community:}
\begin{itemize}
    \item Promotes Tech for Health: It shows that technology can help older people stay healthy and safe.
    \item Sets a Good Example: It sets a good example for how to make new technology for older people.
    \item Saves Money: It can save money on healthcare by keeping older people out of the hospital.
    \item Supports Staying at Home: It helps older people stay in their own homes and communities, which is good for them and for the community.
\end{itemize}
By making ElderCare Connect, we can help older people live better, safer, and happier lives. We can also give their families peace of mind and make our community a better place for older people to live.