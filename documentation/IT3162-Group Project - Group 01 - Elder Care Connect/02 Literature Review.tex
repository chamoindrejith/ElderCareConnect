\newpage
%%%%%%%%%%%%%%%%%%%%%%%%%%%%%%%%%%%%%%%%%%%%%%%%%%%%%%%%%%%%%%%%%%%%%%

%%%%%%%%%%%%%%%%%%%                 literature
\chapter{Literature Review}

\section{Introduction}
As per the recent litearture, aging is a universal phenomenon expected to continue 
existing for the time being. By 2013, approximately 
841 million individuals (11.7\%) had passed the age of 
60 \cite{MH17464}. Although aging cannot be prevented, better understanding by local residents can 
help them feel healthy despite functional deterioration 
and some diseases \cite{eldersHealthFactors}. As individuals age, their intrinsic capacities decline, and the risk of multi-morbidity increases, resulting in the need for ongoing monitoring or treatment \cite{xue2021intrinsic}. \\

An early study shows the factors affecting fall accidents in elderly people who live alone. According to that,\\ 
\begin{itemize}
    \item Main behavioral factors include taking medicine to treat chronic diseases, using alcoholic beverages, not exercising, wearing comfortable shoes, Inappropriate clothing. 
    \item Main environmental regularly using stairs two-story house, split-level house, light switch not nearby, mattress. 
    \item Main economic and social factors, frequently attending social events, missing home visits from the health service unit. 
    \item The main factor was caregiver. The sub-factor is living with offspring and having to live alone.
\end{itemize}
It was found that above factors are affecting the incidence of falls in elderly people who live alone
 \cite{Srenual_Kanokthet_2024}.

Moreover research articles highlight that for many countries, the emergence of an ageing population is fast becoming an increasing public health concern. Healthcare costs are continuously rising and the quality of services does not meet the needs of modern society. Remote real-time health monitoring provides one possible solution to overcoming these challenges. Constantly monitoring the
 health of elderly people via wearable devices and fitness trackers proved different avenues in facilitating Early Intervention Practices EIP and reducing release rates \cite{RemoteHealth}.\\
 
According to the recent studies,  Generally, innovative information and communication technology can play a significant role in caring for elderly patients, at their own homes or at other healthcare environments\cite{MH17464}. Over the past few decades, smartphones have fundamentally changed our daily lives. In terms of health, older adults now have more opportunities to obtain health knowledge and improve the convenience of their medical treatment using apps. Several studies have demonstrated the effectiveness of mobile health interventions in the management of lifestyle factors (eg, improving dietary habits, quitting smoking, and increasing physical activity) and chronic diseases (eg, diabetes, Parkinson disease, high blood pressure) \cite{chinese_survey}.  In 2019, the adoption rate of smartphones by older adults aged 55–91 years was 40–68 \% text, mHealth is a promising tool for promoting healthy aging through evidence-based self-management interventions that help older adults maintain functional ability and independence \cite{Liaw-2019}. The effectiveness of mHealth in promoting healthy behavior and managing chronic diseases has been proven. Usability is considered a vital factor influencing the adoption of mHealth by the elderly which is defined as “the extent to which a system can be used by specified users to achieve specified goals with effectiveness, efficiency and satisfaction in a specified context of use” \cite{eldersHealthFactors}. As mHealth technologies grow, emerging applications of the technologies will enable life-changing uses for the elderly population. The application of mHealth has the potential to improve outcomes and change the course of health care as it is provided today \cite{mHealthforAgingPopulation}.\\

 The growing elderly population, accompanied by the increasing prevalence of chronic diseases associated with aging, will have profound implications for the health care system for decades to come. Therefore, it is becoming essential to engage technologies such as healthcare sensors and wearable with our healthcare systems, in order to have a safer and convenient environment for everyone to live in \cite{RemoteHealth}.
\\

Furthermore there are more projects which has done regarding this title under several categories. Recent researches identifies them as, The project Aware Home uses a wide variety of sensors, covering from specifically designed smart floors to more typical video and ultrasonic  sensors, together with social robots to monitor and help older adults. The Ubiquitous  HomeProject is also centred in residents monitoring through a robot, which includes a dialogue based interface and operates as an intermediate platform between the smart home and the end users. In addition, some projects have focused on the monitoring patients
suffering from a chronic disease, such as CASAS project, which use the smart-homes environment to monitor patients suffering from dementia, which is similar to ENABLE
project with the goal of giving them more autonomy in their lives. While, DOMUS and IMMED projects focus on behaviour recognition for patients suffering from Alzheimer disease. Grenoble Health Smart Home proposed a set of tools to measure patients activity in hospitals via several indicators of their activities in daily living, mobility, and repartition of stays and displacements. In the Gloucester Smart House project, a telehealth system was designed, based on lifestyle monitoring, with the pretension of continuously gathering information about the patients activity during daily routines \cite{RemoteHealth}. \\

In addition to that past literature has identified that Wearable sensors and health-monitoring devices, including heart rate sensors, pulse  sensors, oxygen sensors, and blood pressure sensors, play a crucial role in remote health monitoring systems, whether in open or closed environments, for observing patients. These sensors detect any abnormalities in the patient’s behavior, prompting immediate action from caregivers or doctors, enabling them to take necessary measures promptly to address the situation. In addition to these wearable sensors, vision-based sensors are also employed to monitor the health conditions of patients. For instance, a camera mounted near the patient’s vicinity keeps track of the patient’s movement. If the system detects any abnormal movement by the patient, it promptly triggers an alarm to notify the caretaker. By combining wearable and vision-based sensors, healthcare providers can comprehensively monitor and respond to the well-being of patients in real-time \cite{ASurvey}.\\

Throughout the past literature, a common problem found while researching was the lack of feature rich applications available in Sri Lanka. Though there are applications available with appealing features 
they are mainly available in the other countries. Therefore the aim of this project is to provide an application that both the family and elderly person can interact with enabling elder care.


\newpage