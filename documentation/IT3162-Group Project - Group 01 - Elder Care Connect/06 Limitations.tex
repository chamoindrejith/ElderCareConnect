\chapter{Limitations, Recommendations, and Conclusion}
\section{Introduction}
This section provides a comprehensive overview of the Limitations, Recommendations, and Conclusion of the Elder Care Connect project. As with any system, there are inherent limitations that impact its functionality, ranging from technical constraints to user adoption challenges. Addressing these constraints, we offer recommendations for future enhancements to improve system efficiency, security, and user experience. Finally, we summarize the project’s significance and potential impact on elderly care services, highlighting the value it brings to caregivers and elderly individuals alike.
\section{Limitations}
While the proposed Progressive Web Application (PWA) offers significant advantages for elderly individuals living alone, there are certain limitations that need to be addressed:
\begin{itemize}
    \item \textbf{Internet Dependency:} The real-time health monitoring and emergency alert system require a stable internet connection, which may not always be available in remote areas.
    \item \textbf{Device Compatibility:} The effectiveness of health tracking depends on integration with wearable devices, which may not be universally available or compatible with all elderly users.
    \item \textbf{User Adoption and Digital Literacy:} Elderly individuals with limited technological experience may find it difficult to navigate the application without external assistance.
    \item \textbf{Privacy and Security Concerns:} Handling sensitive health data requires robust encryption and authentication mechanisms to ensure data security and compliance with healthcare regulations.
    \item \textbf{Healthcare Integration Challenges:} The effectiveness of the app depends on cooperation from healthcare providers, which may vary based on location and medical policies.
\end{itemize}
\section{Recommendations}
To address these limitations and enhance the application's usability, the following recommendations are proposed:
\begin{itemize}
    \item \textbf{Offline Functionality:} Implement local storage and caching to enable essential features such as medication reminders and emergency contact access even in offline mode.
    \item \textbf{Improved Device Support:} Expand compatibility with a wide range of wearable health devices to accommodate diverse user needs.
    \item \textbf{User Training and Support:} Develop easy-to-follow tutorials and provide customer support services to help elderly users and caregivers navigate the application.
    \item \textbf{Enhanced Security Measures:} Strengthen encryption techniques, implement multi-factor authentication, and ensure compliance with healthcare data regulations.
    \item \textbf{Healthcare Partnerships:} Collaborate with medical institutions to facilitate seamless integration with healthcare services and ensure accurate monitoring of user health data.
    \item \textbf{Customization Options:} Allow caregivers and family members to personalize alerts, notifications, and interface settings to better suit individual user needs.
\end{itemize}
\section{Conclusion}
The development of the ElderCare Connect application aims to enhance the safety, health monitoring, and social well-being of elderly individuals living alone. By leveraging real-time health tracking, emergency alerts, and social engagement features, the application ensures improved care and independent living for elderly users. The integration of wearable devices and Progressive Web Application (PWA) technology provides accessibility across multiple devices while maintaining offline support.\\
Despite certain limitations such as internet dependency and healthcare integration challenges, the recommended solutions—offline capabilities, improved device support, and enhanced security measures—can significantly enhance the application's effectiveness. With further development and collaboration with healthcare providers, this application has the potential to revolutionize elderly care, offering peace of mind to families and caregivers while promoting independent living for elderly individuals.



