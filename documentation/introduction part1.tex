
\documentclass{article}
\usepackage{geometry}
\geometry{a4paper, margin=1in}
\usepackage{amsmath}
\usepackage{amsfonts}
\usepackage{amssymb}

\title{Introduction}


\begin{document}

\maketitle

\section*{Introduction}

Elderly individuals who live alone face significant health and safety challenges, including medical emergencies, falls, and feelings of isolation. Due to the lack of immediate assistance, these incidents can lead to delayed medical attention and complications. Furthermore, isolation can negatively affect their mental and emotional health. The absence of regular check-ins and health tracking increases the risk of undetected health deterioration.

The project aims to develop a Progressive Web Application (PWA) designed to enhance the safety, health monitoring, and social well-being of elderly individuals living alone. Through tracking real-time health (using APIs like Google Fit), the app will ensure caregivers/family members can monitor their health status from a distance.

The ElderCare Connect will function across multiple devices (desktops, smartphones, and tablets) with native app-like experiences, providing real-time health monitoring, emergency alerts, and social engagement features. Push notifications will ensure immediate attention during emergencies. Messaging features will help elderly connected with family, friends, and caregivers, mitigating loneliness and providing emotional support with PWA advantages. Data security and privacy are top priorities.

\subsection*{1.1 Background}

In recent decades, smartphones have significantly changed the way we live. In terms of the health sector, elderly individuals now have greater access to health information and can enhance the convenience of managing their medical care through mobile applications. Several studies have shown the positive impact of mobile health interventions in managing lifestyle factors such as improving diet, quitting smoking, and increasing physical activity, as well as controlling chronic conditions like diabetes and hypertension. This change has also influenced the relationship between patients and their general practitioners.

Recent research has shown that patients see mobile health apps as valuable tools for self-monitoring and self-managing their own health. However, they also acknowledge certain limitations, such as issues with battery life and concerns regarding the legal aspects of electronic prescribing.

Moreover, there has been increasing concern regarding the rising number of elderly individuals, particularly those living alone due to the population ageing across the globe. This demographic group faces great challenges, including high health risks, social disconnectedness, and slow responses in case of emergencies that may endanger their lives. Considering that such a severe future scenario would likely occur much faster than a call for help or even faster than some normal medical conditions, there is a need for solutions that can protect the elderly, particularly those without nearby family members, from possible future threats.

\subsection*{1.2 Problem Statement}

Older people who live by themselves can have a hard time. If something bad happens, like a fall or a sudden health problem, they might not get help right away. This can make things worse for them. If they fall and can't get up, they could get hurt badly. If they have a disease like diabetes, they need to watch their health closely. If they don't, they could have big problems.

Feeling alone is also a big worry. When older people don't see other people, they can feel sad. This can make it hard for them to think clearly and remember things. They might not want to go to the doctor or take their medicine.

Even though there are phone apps that can help, they don't always work well for older people. Many apps only do one thing, like check your health. They don't have everything older people need in one place. They might not have a way to call for help or talk to family. And they are often hard to use because the buttons are small and the words are confusing.

That's why we need a better app that is made just for older people who live alone. It needs to watch their health, call for help when they need it, and let them talk to family and friends.

ElderCare Connect will be this kind of app. It will use special sensors to watch their health and send alerts if something is wrong. It will have big, easy-to-see buttons and words. It will work even if they don't have internet all the time.

By making this app, we can help older people live safer, healthier, and happier lives. Their families can also feel better knowing they are okay.

\subsection*{1.3 Main Objective}

Our main goal is to make ElderCare Connect. It's a special app that will help older people who live by themselves. We want to make them safer, watch their health, and help them feel connected to others.

Here's what we want ElderCare Connect to do:

Keep them safe:

\begin{itemize}
    \item Let them call for help fast if something bad happens. They can push a button to call family or emergency services.
    \item Let family see where they are. This way, if they need help, family can find them easily.
    \item Remind them to take their medicine. It's easy to forget, so the app will send reminders.
\end{itemize}

Watch their health:

\begin{itemize}
    \item Use special devices to check their heart rate, blood pressure, and how much they move around.
    \item Let family see their health information. This way, they can watch for problems.
    \item Give them tips on how to stay healthy. The app can give advice based on their health information.
\end{itemize}

Help them feel connected:

\begin{itemize}
    \item Let them talk to family and friends easily. They can send messages and share pictures.
    \item Help them find things to do in their community. They can find groups and events to join.
    \item Let them share their stories and memories. This helps them stay connected to loved ones.
\end{itemize}

We also want to make sure the app is:

\begin{itemize}
    \item Easy to use, with big words and simple buttons.
    \item Works even if they don't have internet all the time.
    \item Safe and secure, so their health information stays private.
\end{itemize}

If we do all of this, ElderCare Connect can:

\begin{itemize}
    \item Help older people live better lives.
    \item Keep them out of the hospital.
    \item Make them feel happier and less lonely.
    \item Give their families peace of mind.
\end{itemize}

\subsubsection*{1.3.1 Specific Objectives}

To make ElderCare Connect the best it can be, we have some specific things we want to achieve:

Check Health:

\begin{itemize}
    \item Connect to devices like smartwatches to see heart rate and activity.
    \item Make a simple page to show all the important health numbers.
    \item Create a way to find problems in the health data and send alerts.
    \item Keep track of health history so people can see how things change over time.
\end{itemize}

Get Help Fast:

\begin{itemize}
    \item Make a button to send alerts to family or emergency services.
    \item Create a list of emergency contacts that is easy to find.
    \item Make the app know if someone falls and send an alert automatically.
    \item Use GPS to find the person's location if they need help.
\end{itemize}

Stay Connected:

\begin{itemize}
    \item Let people send messages and share their location with family.
    \item Create a way to connect with friends and other people in the community.
    \item Let people share photos and videos with their loved ones.
    \item Show a calendar of events happening nearby so people can find things to do.
\end{itemize}

Easy to Use:

\begin{itemize}
    \item Make the app simple with big words and easy buttons.
    \item Let people use their voice to control the app.
    \item Let people change the settings to make the app work best for them.
    \item Give instructions on how to use the app.
\end{itemize}

Works Offline:

\begin{itemize}
    \item Make sure the app works even without internet.
    \item Save health data on the device so it can be seen offline.
    \item Make sure data updates when the internet comes back on.
    \item Let people use important features even when they are not connected to the internet.
\end{itemize}

Keep Data Safe:

\begin{itemize}
    \item Use strong passwords and ways to keep information secret.
    \item Only let people who are allowed see the health data.
    \item Have a plan to back up data in case something goes wrong.
    \item Follow all the rules about keeping health information private.
\end{itemize}

\subsection*{1.4 Scope of the Study}

When we talk about the "scope" of our project, we mean what ElderCare Connect will do and who it will help. It's important to know what we will focus on and what we won't do so we can make the best app possible.

Who We're Helping:

\begin{itemize}
    \item We're making this app for older people who can mostly take care of themselves but need some help staying safe and healthy.
    \item They should know how to use simple technology, like phones and tablets.
    \item We're not making this for people who live in nursing homes or need a lot of medical care.
\end{itemize}

What the App Will Do:

\begin{itemize}
    \item It will watch their health using devices they wear.
    \item It will send alerts if they need help in an emergency.
    \item It will help them talk to family and friends so they don't feel lonely.
    \item It will work on phones, tablets, and computers.
    \item It will work even if they don't have internet all the time.
    \item It will keep their health information safe.
\end{itemize}

What Technology We'll Use:

\begin{itemize}
    \item We'll use the latest computer languages to build the app.
    \item We'll use devices like smartwatches to check their health.
    \item We'll use a safe and reliable place on the internet to store their information.
    \item We'll follow all the rules to keep their data safe and private.
\end{itemize}

Where We'll Use the App:

\begin{itemize}
    \item We'll start by testing the app in one area to make sure it works well.
    \item We'll make sure the app can be used in different languages and cultures.
\end{itemize}

How Long It Will Take:

\begin{itemize}
    \item We have a plan with dates for when we'll finish each part of the app.
    \item We'll focus on making the first version of the app, but we plan to add more features later.
\end{itemize}

What We're NOT Doing:

\begin{itemize}
    \item We're not making new medical devices.
    \item We're not giving medical care directly.
    \item We're not making a new social media site.
\end{itemize}

\subsection*{1.5 Significance of the Study}

ElderCare Connect is important because it can make a big difference in the lives of older people, their families, and the community. It can help older people stay healthy, safe, and connected, and it can give their families peace of mind.

How It Helps Older People:

\begin{itemize}
    \item Better Life: It helps them live a better life by watching their health, getting help fast when they need it, and talking to family and friends.
    \item Less Time in the Hospital: It can keep them from having to go to the hospital by finding problems early and helping them get help fast.
    \item Happier Feelings: It helps them feel less lonely and more connected, which makes them happier.
    \item Stay Independent: It helps them stay in their own homes longer because they have the support they need to stay safe and healthy.
\end{itemize}

How It Helps Families:

\begin{itemize}
    \item Peace of Mind: It gives families peace of mind knowing that their loved ones are safe and healthy.
    \item Fast Help: It makes sure they get help fast if something goes wrong.
    \item Good Communication: It helps them talk to their loved ones and stay connected.
    \item Less Worry: It takes some of the worry away because they know their loved ones have the tools they need to stay safe.
\end{itemize}

How It Helps the Community:

\begin{itemize}
    \item Promotes Tech for Health: It shows that technology can help older people stay healthy and safe.
    \item Sets a Good Example: It sets a good example for how to make new technology for older people.
    \item Saves Money: It can save money on healthcare by keeping older people out of the hospital.
    \item Supports Staying at Home: It helps older people stay in their own homes and communities, which is good for them and for the community.
\end{itemize}

By making ElderCare Connect, we can help older people live better, safer, and happier lives. We can also give their families peace of mind and make our community a better place for older people to live.

\end{document}


